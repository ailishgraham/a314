Random info file 

There are two types of epidemogical studies used: time-series for short-lived high pollution events and cross-sectional cohort studies for exposure to background PM concentrations over longer periods \cite{Heal2012}. These studies have found that, of all air pollutants, PM$_{2.5}$ is associated with the worst impacts on health as a result of their sufficiently small aerodynamic diameter allowing them to be respired deep into the lungs. The Committee on the Medical Effects of Air Pollution (COMEAP) estimate that in the UK cardiopulmanary mortality due to long-term exposure to PM$_{2.5}$ is  9\% per 10 ${\mu}$gm$^{-3}$ (all cause: 6\% per 10 ${\mu}$gm$^{-3}$) and that the association between hospital admissions for cardiovascular problems and daily PM$_{2.5}$ was 1.4\% per 10 ${\mu}$gm$^{-3}$ \cite{Comeap}. Other work has also found a clear link between increased daily concentrations of PM$_{2.5}$ over a short time period of a few days and increases in hospital admissions and mortality \cite{Atkinson2015}. Work by COMEAP also identified that there is no safe level of exposure to PM pollutants and there is also a lack of understanding regarding the effect of different species on mortality and other health impacts \cite{Coemap}. \\